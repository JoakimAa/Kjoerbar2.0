\section{Implementasjon}
Lenke til GitHub: https://github.com/JoakimAa/Kjoerbar2.0

Så langt i prosjektet har vi implementert basis funksjonalitet for appen. Dette inkluderer en innloggingsskjerm med integrasjon mot Firestore og navigasjon rundt i appen. Vi har inndelt appen i to aktiviteter: SignInActivity og MainActivity, hvor MainActivity er der begge komponentene for literal-navigation ligger, og inneholder en navHostFragment. 

\subsection{Database}
Firestore er implementert til å behandle innlogging. Ved første oppstart vil appen vise en innloggingsskjerm hvor brukeren kan velge å logge inn med en google-bruker eller med en e-post. Innlogging via google fungerer på samme måte som når man logger inn i nettleseren, mens e-post innloggingen fungerer litt annerledes. Hvis e-posten man skriver inn ikke er registrert i databasen, vil man få valg om å opprette en ny bruker. For øvrig kan man ikke slette brukeren sin, men dette er noe vi tenker å implementere senere.

I databasen lagrer den imidlertid data for alder, navn, vekt, kjønn og høyde.

\subsection{Bugs}
Dessverre er det en del bugs fremdeles, spesielt nå etter at vi nettopp har oppdatert bottom- og drawer-navigation. Det er noen problemer med backstacken, siden tilbake-pilen alltid navigerer brukeren tilbake til økt-siden (Startsiden).
