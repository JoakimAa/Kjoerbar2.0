\section{Arbeidsprosess}
Arbeidsprosessen har bestått av mye planlegging i forkant av prosjekttiden. Vi begynte med å tenke hva brukergruppen forventer av appen vår, og skrev tilsvarende krav. Videre har vi brukt Trello til å holde styr på hvilke krav som er tilfredstilt, og vi har brukt veiledningstimer som feedback på appen. Siden brukergruppen vår er studenter, har vi hatt mye dialog med medstudenter rundt appen og hørt på deres meninger. På denne måten har arbeidsprosess bestått av en syklus av kontinuerlig feedback og og arbeid. Når det er sagt har det også vært perioder hvor vi ikke har jobbet med prosjektet.

Angående GitHub, har vi vært nøye med grenstrukturen. Vi bruker tre hovedgrener med navn \textit{doc}, \textit{development} og \textit{main} som har vært sin rolle. development er grenen som all kode lastes opp til. Det er ingen garanti på at prototypen som ligger i development-grenen er stabil, siden det er her all hyppig utvikling pågår. Vi har derimot prøvd å være konsekvente med koden som ligger i main-grenen, siden her forventes det at prototypen er komplett og kjører uten feil. Til slutt brukes doc-grenen når vi skriver på prosjektrapporten.