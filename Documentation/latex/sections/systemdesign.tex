\section{Systemdesign}
\subsection{Navigasjon}
Appen består av to hovedelementer for navigasjon, et bottom-navigation element, og et navigation-drawer element. I tillegg til disse har den en swipe-funksjon når man befinner seg i forsiden, kartsiden og vennesiden. Det vil være mulig å navigere mellom disse tre sidene med bottom-navigation elementet, men også å swipe fra venstre til høyre og motsatt. Tilgjengelig fra de fleste sider vil være en navigation-drawer som brukes til å navigere til resten av appen.

Grunnen til denne oppdelingen er for å skille mellom sider som blir brukt mye og sider som blir brukt lite. Vi har valgt å bruke bottom-navigation og swipe-funksjonen til forsiden, kartsiden og vennesiden, fordi de er sentrale og brukes nesten hver gang appen er i bruk. Forsiden er det samme som oversiktssiden, så dette vil være landingssiden når brukeren logger inn. I navigation-drawer elementet kan man blant annet navigere seg til innstillinger og enhetskatalogen.

\subsubsection{Flyt}

\begin{figure}[H]
    \centering
    \includegraphics[scale=0.5]{images/lille_promille_float_diagram.drawio.png}
    \caption{Skisse som viser hvor man kan navigere seg til fra hvilke sider i appen.}
\end{figure}

\subsubsection{Navigation Pattern}
Bla bla bla

\subsection{Teknisk oppbygging}
Blablabla

\subsubsection{Lagring og uthenting av data}
For å lagre brukerdata benyttes Firestore. Dette gjøres via Firebase, her lagres brukere i en "user" collection. Hver bruker lagres med en UUID som autogeneres fra Firebase. 

Brukerinformasjonen lagres rett i dokumentet til brukeren, mens enhetskatalogen og økt historikken lagres i collections på brukeren. 
Disse 

\subsubsection{Brukerhåndtering}
Håndtering av brukere gjøres gjennom Firebase.

\subsection{Ressurser}
\subsubsection{Services}


\subsubsection{Fragmenter}

\subsubsection{Bilder}
Bilder i appen vil bestå av ikoner, logoer, emojier og eventuelt profilbilder. Ikonene brukes til å veilede brukeren der det er raskere å oppfatte ikonet enn å det er å lese teksten, for eksempel til navigasjonspunktene i bottom-navigation og navigation-drawer. Ikonene er en blanding av lånte og egenskapte ikoner som følger design prinsippene til Material Design dokumentasjonen(1). Logoen lager vi selv, og vi skal ha variasjoner av logoen 

\subsubsection{Biblioteker}