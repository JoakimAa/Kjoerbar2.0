\section{Hva ble vi ikke ferdig med}
\subsection{Vennelisten}
Noe vi var svært engasjerte med i starten av prosjektet var å implementere en venneliste, der man kunne legge til venner og sende direktemeldinger til hverandre. Vennelisten skulle være en sentral funksjon som vekker interesse i appen. Det skulle også være en funksjon for å se andre venner sin drikkestatus vist som en \textit{emoji} ved siden av brukernavnet.

\subsection{Kartet}
I den ferdige appen rakk vi å implementere kartet via google sitt API, men det ender dessverre der. Fremdeles mangler det at appen lagrer GPS-posisjoner til hver økt, slik at brukeren kan vite hvor de har vært gjennom økten. Idéen var at hver gang brukeren la til en drikke i økten, så skulle den lagre posisjonen i det drikken ble lagt til, slik at man til slutt kunne se hvor man har gått i løpet av økten. Vi var også inne på tanken å implementere vennelisten i kartet ved å vise posisjonen til vennene sine, tilsvarende "Snap-Map" fra SnapChat-appen.

\subsection{Drikkegrenser}
En annen ting som ville økt brukeropplevelsen var en drikkegrense feature i øktsiden. Dette gir brukeren mulighet for å sette av en grense på hvor mye de ønsker å drikke. På den måten ville appen hjulpet med å regulere inntaket alkohol, slik at brukeren får mer kontroll over seg selv. Det er for eksempel nyttig dersom brukeren ønsker å være edru til et visst tidspunkt, og da kan være bevisst på når de kan kjøre bil. Drikkegrensen består av tre forskjellige typer grenser som brukeren kunne velge:

\subsection{Varsler}
En annen hovedfunksjon var at appen skulle varsle brukeren ved spesifikke situasjoner. Blant annet skulle den varsle når brukeren bør slutte å drikke, eller hvis brukeren beveger seg for fort under en økt, slik at appen mistenker at brukeren fyllekjører. Det er opprinnelig denne idéen som er grunnen bak kodenavnet til appen "Kjørbar", siden appen ville i starten ha som funksjon å måle når man kan kjøre bil.

\subsection{Scanne etiketter på flasker}
En av visjonene vi hadde i planleggingsfasen var at appen skulle ha en funskjon som tillot brukeren å scanne etiketter på flasker med kamera på mobilen, slik at nøyaktig denne drikken ville blitt tilgjengelig for brukeren å legge til i økten eller drikkekatalogen. Dette ville hjulpet mye, siden da hadde det vært lettere å legge til nye drikker, særlig mens man er påvirket.